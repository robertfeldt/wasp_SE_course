\documentclass{article}
\usepackage{setup/mypackage}
\newacronym{ai}{AI}{Artificial Intelligence}
\newacronym{rl}{RL}{Reinforcement Learning}
\newacronym{xai}{XAI}{Explainable \glsentrylong{ai}}
\newacronym{xrl}{XRL}{Explainable \glsentrylong{rl}}


\title{WASP Software Engineering\\Assignment}
\author{Franco Ruggeri}

\begin{document}

\maketitle

\section{Introduction}

My research is focused on \gls{xrl} applied to telecommunication networks.

\Gls{rl} is a branch of \gls{ml} that involves training an agent to act optimally in an environment to optimize a given reward signal. Given its closed-loop framework, \gls{rl} is a promising solution for automating the optimization of large 5G networks, which involves tuning a large number of interconnected parameters. This optimization is not feasible to perform manually by human operators given the scale of such systems. Today's commercial solutions rely on rule-based systems that are rigid and suboptimal. \Gls{rl} can instead adapt to dynamic network conditions and optimize performance in real-time.

However, network automation comes with risks, as wrong parameter adjustments can lead to degraded performance and services for end users. For this reason, network operators seek transparency and reliability in such solutions. State-of-the-art \gls{rl}, similarly to other branches of \gls{ml}, is notably opaque and lacks the required transparency to be adopted in production environments.

The goal of \gls{xrl} is to design methods that enhance the interpretability of \gls{rl} agents so that humans can understand and trust \gls{rl}-based decisions.

% further topics in case of need for more content:
% - reasons for lack of interpretability in RL
% - post-hoc vs intrinsic XRL

\section{Lecture Principles}

\section{Guest-Lecture Principles}

\section{Data Scientists versus Software Engineers}

\section{Paper Analysis}

\section{Research Ethics \& Synthesis Reflection}

\end{document}
