The two concepts from the guest lecture that resonated with me and my research broadly relate to the idea that planning steps systematically and in a disciplined manner before starting a project is beneficial. Although my research area is not closely linked to software engineering, I found that requirements engineering plays a role even in my work, particularly regarding the following two concepts:\par\vspace{1em}
\noindent
\textbf{1.} \underline{Identifying stakeholders:} The stakeholders in my research are heterogeneous. For example, when building simulations to compare different methods of doing statistical hypothesis testing, the list of people who have a certain interest in the “system” (simulations) to be developed includes other PhD students, supervisors I work with regularly, professors unfamiliar with the system's structure, and domain experts who lack mathematical backgrounds but are proficient in the scientific field where my research applies (e.g. gait analysis, as stated in the introduction). Listing all the stakeholders for a given project allows to reduce the risk of missing important needs or constraints and helps prioritize certain requirements over others. For example, imagine that the output of the simulations is displayed through an interactive dashboard. Through this dashboard, a user/stakeholder may change some model settings (the variability of the synthetic data, the metric used to check if the data are coherent with the null hypothesis $H_0$, the number of functions/time series in the dataset, etc.). Since there are many settings that could be changed, it is desirable to keep the dashboard as simple and readable as possible by excluding unnecessary elements. The interest of the stakeholders can clearly help decide which simulation settings to include in the dashboard. A domain expert might not be interested in changing the number of time series in the dataset, while other stakeholders with a mathematical background might want to use this simulation setting to see the asymptotic properties of the proposed model.\par\vspace{0.7em}
\noindent
\textbf{2.} \underline{Cost of defect removal, goal modelling and refinement:} In my research, the cost of defect removal is directly linked to the fact that simulations are computationally expensive: the generated time series are sampled at many points, producing very long vectors. If the setup of the simulations turns out to be wrong (for example, how synthetic data are generated, or how null and alternative hypotheses are formulated), fixing that mistake means re-running everything. This is costly both in terms of computation and time, similar to how software engineering defects become more expensive to fix the later they are discovered in the development pipeline. By carefully checking and refining requirements early (e.g., making sure the simulation setup is correct before running large-scale experiments), it is possible to decrease the waste in resources and time. A systematic goal refinement step prior to the implementation of the simulations’ pipeline also helps make the implementation process smoother. This also removes redundancy: if the goal is precisely formulated (e.g. “we want to understand how the variability in the original dataset influences the model”), then it is possible to write more concise code that could, possibly, be updated and expanded if/when new goals are formulated.
