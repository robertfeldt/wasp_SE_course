The authors of this paper (\cite{Pinto}) discuss contextualized coding assistants. When making use of general purpose AI chatbots to write code, some issues might be encountered such as a lack of specificity of the generated code. On the contrary, contextualized coding assistants can be used to generate text and code that better fits some specific requirements. These "specialized" chatbots are based on the Retrieval-Augmented-Generation (RAG) methodology: when prompting a contextualized LLM, the RAG system allows the model to "look for" the answer to the prompt by retrieving the relevant information from a given database. Essentially, contextualized AI assistants work by firstly retrieving information from the database and secondly by generating text or code in response to a given prompt. The database contains the "context" that the AI assistant uses to better taylor the answer to the prompt to the needs of the company, the research group or any other type of user. The topics discussed in this paper closely relate to the topics of the lectures and, more broadly, to the engineering of AI systems, where code needs to follow specific legal requirements, numerous standards or rules and is checked both statically and dynamically. Using AI chatbots can reduce human errors in the code and using \underline{specialized} AI chatbots helps developers navigate the (sometimes) overwhelming number of rules and constrictions that need to be followed when writing code, many of which can be difficult to manage by humans alone.\\
The authors of the paper conducted a study to qualitatively understand if the use of specialized chatbots in a company can be beneficial to software developers. Moreover, they investigated whether such AI tools can easily be integrated in the workflow of the company's teams. The contextualized chatbot analyzed in the paper is called "StackSpot AI" and its influence in the workflow was tested on a group of employees of a software company. In particular the database accessed by the RAG contained exemplary code snippets, guidelines regarding repository commits, a list of software requirements, and other information.%\par\vspace{0.7em}

The study proposed by the authors of the paper mainly consisted with a demonstration of the capabilities of StackSpot AI to the developers who participated in the study. Then, the participants were allowed to explore StackSpot AI by solving some tasks and interacting with the database accessed by the retrieval system of the AI assistant. The study ended with a group discussion. The results of the survey demonstrated some benefits of contextualized AI coding and a general perception of increased productivity. This was due to the ability of the chatbot to quickly generate precise and relevant code snippets, that satisfied the company's code standards and requirements. Moreover, StackSpot AI works as a chatbot, i.e. it allows users to interact and iteratively refine the provided code. On the other hand, some challenges were highlighted: some users found it challenging to understand which information to include in the database accessed by the chatbot's RAG system. Indeed, it appeared that a higher amount of excessively specific information contained in the database, would affect negatively the quality of the answers, which would require much more refinement and adjusting.%\par\vspace{0.7em}

It is challenging to describe a larger AI-intensive software project where the paper’s ideas could be of "practical" use. This is because the paper does not provide a potential solution to a practical problem (as, instead, in Paper 2, see later). Rather, the paper gives an overview of the benefits/challenges of integrating contextualized AI coding assistants in a software developer team, that might instead be used to working with general-purpose AI coding assistants. At the same time, in any large AI-centered software product, the use of chatbots to generate code is becoming more and more frequent. Each project's needs and requirements are unique, hence the use of ad-hoc coding could be highly beneficial. In my own research, it is desirable that the R code written to test hypothesis or cluster functional data adheres to certain standards that could make the code eligible to be potentially published on CRAN (\cite{cran}, which collects publicly-available R packages) in the future. At the same time, it is often the case that new research builds up on top of previous results. In these cases, it is desirable that code design choices implemented by previous researches are maintained, to facilitate future researchers who will approach the same topic. To find a trade-off between all these (possibly conflicting) requirements, it could be beneficial to adapt a contextualized AI coding assistant by including in its database exemplary code snippets and all the necessary requirements. This is one way I could change my research project so that, over time, it better supports the paper’s AI idea.
