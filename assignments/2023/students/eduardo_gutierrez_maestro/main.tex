% \documentclass{article}
\documentclass[11pt]{article}
\usepackage[a4paper, total={6in, 8in}]{geometry}

\usepackage{titling} % Required to customize the title
\setlength{\droptitle}{-3cm} % Adjust the value as needed

\title{Software Engineering assignment}
\author{Eduardo Gutierrez Maestro}
\date{June 2023}

\begin{document}
\maketitle

\begin{abstract}
    This document contains the text in relation to the assignment of the second module of the WASP course "Cloud Computing and Software Engineering". Reflections from several topics discussed during the course will be discussed in this document, highlighting always similarities and potential use cases with my own research project. Finally, a conclusion and reflection based on my point of view about where this field is moving will be given. 
\end{abstract}
% -------------------
My research project is named: Artificial Intelligence for human well-being understanding. The main goal of this project is to use technology to enhance the life quality of people during their daily life activities. This project was part of a higher project where it combined robotic platforms with ambient and wearable technology with the aim of creating a system capable of understanding the moods/emotional states of people during their daily activities. The whole system will act as a smart platform contributing to the progress of active ambient living technology.
I started my research by using wearable sensors. In concrete, I am using a smartwatch that collects physiological-based data. My goal is to find new ways to connect affective states, i.e. moods such as happiness, sadness, or stress, with the data our body generates in an intelligent and automatic way. Initially, I started working on a dataset consisting of 4 participants monitored all day throughout 15 days. The participants reported their moods through a smartphone app, that recorded their level of valence and arousal for later interpolation into one mood or affective state.
The experiments conducted during this project showed the numerous challenges presented by this problem. One of them is the little amount of quality data available to the public. Another challenge encountered is the high uncertainty presented in the discrete labels collected with the questionnaires. This uncertainty and lack of control over the verification of these labels make the learning process a hard task, especially when developing a machine learning algorithm. Finally, there is a high variability between subjects, that is, the different responses recorded in the data between different participants make the classification task quite challenging to implement generalizable models. Some of the solutions to tackle this problem is: changing the label learning paradigm used in this first project, that is, finding a way to convert each of the discrete labels into a distribution of labels of the possible total set. This label would be more informative and guide the model through the learning step to more accurate and explainable solutions. Moreover, this variability may help to solve the problem of differences between participants, thus, making these types of systems more generalizable. 

Throughout the assignment, I will put examples of a product that is the result of the success of the research project I am working on. This is purely for clarification of the topics I will discuss.


% -------------------
From the lectures I can highlight some concepts and discussions that got my attention and that I could relate to my research. Given my research topic and project motivation, dealing with systems that collect, process, and infer emotional states are extremely dedicated from several points of view, as for example the product development and the ethical aspect. 
One important aspect discussed during the lectures was the quality assurance and model testing of these types of machine learning algorithms. In this concrete system example, this part was quite interesting and appealing to me. The validation and verification steps are of quite importance in these types of systems. 
I will start by commenting on the verification step. This is connected to the development of code to implement the system, which is connected to another part of the lecture where we discussed good practices in Software Engineering, especially in code development. Systems that deal with problems such as classifying mental states are prone to be audited or inspected from time to time if there is a commercial product out of it. For this reason, the coding step can be crucial. For my research project, following good practices in coding such as easy-to-change~(ETC), reversibility or keeping knowledge in plaintext is important. As an example related to my research, imagine that a client of this product fell into depression because the system consistently prompted that his/her mood was bad, or even worse commit suicide. This would alarm the authorities requesting an inspection of the product. Explanations of the code and the overall system should be done. For this reason, having code ready to debug, comments and documentation about the decisions made during the development, are important.
This reflection is directly connected to the verification step mentioned previously. But in the case of the proposed system, verification is about checking whether the system is able to predict a mental state given a set of sensorial input data. 
There is another part discussed during the lecture on quality assurance and testing, which is validation. In this case, this step is about matching the functionality of the system to the end-user needs. This research project is multidisciplinary, and for that reason, collaboration with psychologists is needed. I can relate this validation step discussed during this lecture with a set of interviews with subjects used in the pilot phase of the product development to double-check that the functionality is covered by the system even in extraordinary or outlier cases. 

% -------------------
Out of all the guest lectures given during the course, the one imparted by Fabio Calefato was the most interesting and related to my own research project. Sentiment analysis and emotion recognition are sub-fields within the field of Affective Computing. This field is quite challenging and subjective, and I agree with his statement that there are needed tools specialised in Software Engineering~(SE) for this type of application. Since data drives the success of these models, the use of data used to train models in different domains may bias or confuse solutions given to the targeted SE problem. For example, let's say that I would like to include skin conductivity or electrodermal activity as part of my input data for my model to measure the levels of stress of software developers. This data may differ from others taken with measuring the levels of stress in other domains or contexts, such as for example while driving a car. 
The second concept that I found quite interesting from this guest lecture was \textit{AutoML}. We did a practical session where we trained a sentiment analysis classifier based on textual data. The concept of AutoML was a bit unfamiliar to me, but after the lecture, I can reflect on its usefulness in my research. The development of tools that automatically process, prepares data, give information or even classify or predict some information about the emotional state of the subject using a system, can be valuable for many companies that do not allow the resources to incorporate in their business. This part was quite interesting and moreover, the platform used, DataRobot, may serve me in my research for demos or results demonstrations. 

% -------------------
The topics that are more interesting and related to my research project are human-computer interaction~(HCI) and behavioural software engineering~(BSE). These two topics are related to my thesis in that a system being able to capture, detect and predict emotions, stress or other mental states, can enhance the current systems and products to improve the efficiency and productivity of software. 
My understanding of the term \textit{human-computer interaction} refers to the ability to create and develop technology that provides the end-user with a grateful and pleasant experience. One way to increase productivity and open the development of software to a wider audience is through \textit{toolkits}~\cite{hci-toolkit}. These are packages that ease the software development work. In the last decades, there has been an increasing demand for software developers, and companies want to automatize and include every time more software in their businesses. The learning curve for developing software is slow and can be frustrating. For this reason, embedding tools that are able to detect our frustration, stress levels, or any negative state can make the experience of creating software much more rewarding. The research project I am working on somehow falls within the community of human-computer interaction. The ability of a machine to detect, interpret or simulate emotions or affective states can make the experience of interaction with any computer system much better. One of the data modalities that are used currently in research and many businesses is eye-tracking information. The patterns that we generate while staring at a screen can give information about the cognitive workload, mental fatigue or stress levels. This information could be extremely useful when it comes to the design of toolkits or platforms to develop software. Coding is frustrating when getting stuck in a bug, a system that is able to detect these conditions can inform the corresponding manager by reporting weekly mental well-being of each of the team members. This is connected to the following topic, where focusing more on the human side than the rational or technical side of developing software is more important.
After this course, I have a better understanding of the term behavioural software engineering. When we were introduced to the term software engineering, there was a slide where it summarised it as \textit{"software engineering = computer science  + real-world"}. The part corresponding to the real world is where BSE takes place. Behavioural software engineering is a discipline that aims to address human problems within the field of software engineering. It is a field that tries to focus not that much on the end product but on the individual, groups or organizations that develop such a product. I personally found this term quite interesting, since instead of trying to improve or find new ways to improve the development of software from a technical, and rational point of view, it is focused more on the human beings in charge of developing such solutions. In the literature, I found an interesting approach that highlights the importance of including human values in the production of software~\cite{bse-social}. In such a work, they extract those values by mental representations on three levels: system, personal and instantiation. 
Based on what I read, and what we have explained in the lectures, BSE has a lot of synergy with my own research project. Even though my research is mostly focused on the monitoring of emotional states based on wearable devices, it can be extrapolated to other data modalities such as eye tracking, or typing patterns. Having such a system could enhance current approaches or methodologies within the BSE field. There are three units of analysis in BSE: individual, group and organisation. My research project could take part in the first one, by analysing the well-being of the human beings developing software. Moreover, emotional monitoring systems could be used not only in the individual scope but also in groups of individuals by analysing and correlating the individual information into a well-being score within the developing team.
I think it is still soon to think of the actual implementation of such systems for the use case of software development. There is still young research on how to make emotional monitoring systems robust, explainable and trustworthy. Additionally, I believe that the integration of those in the current methodologies adopted in the field of Software Engineering is a slow process. Data privacy and ethical concerns should be taken into account before embedding such systems.

% -------------------
In this course, I learnt the basic components that define the field of Software Engineering. This topic is outside the field of my research, however, I could find several points where my research could contribute towards the progress of the methods used in the development of software. I would vote that in the near future of the SE research field, more and more systems would focus and centre resources around humans. In my opinion, the concept of Behavioural SE will get more importance over the coming years, moreover, when the development of systems that are able to monitor mental well-being gets more affordable or open to a wider audience of companies. But as discussed earlier, the integration of such systems will be a slow process due to all the regulation steps needed.


\begin{thebibliography}{}
\bibitem{hci-toolkit} Ledo, David, et al. "Evaluation strategies for HCI toolkit research." Proceedings of the 2018 CHI Conference on Human Factors in Computing Systems. 2018.
\bibitem{bse-social} Winter, Emily, Steve Forshaw, and Maria Angela Ferrario. "Measuring human values in software engineering." Proceedings of the 12th ACM/IEEE international symposium on empirical software engineering and measurement. 2018.

\end{thebibliography}
\end{document}
